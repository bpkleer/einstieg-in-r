Liebe:r \texttt{[students-name]},
du hast dich in flexnow oder per E-Mail bei mir im Kurs Statistik+: Einstieg in R leicht gemacht erfolgreich angemeldet. In dieser E-Mail gibt es Hinweise und Anweisungen, damit wir Donnerstag direkt starten können. Bitte nimm \textbf{dir kurz 5 Minuten} und \textbf{lies dir die E-Mail in Ruhe und mit Bedacht} durch.

\begin{enumerate}
	\item Wir beginnen Donnerstag s.t.! Sei bitte pünktlich, damit wir um 16 Uhr direkt beginnen können.
	\item \textbf{RStudio Cloud}: Ihr könnt euch mit eurer JLU-Kennung auf folgender Seite bei RStudio Cloud einloggen: \texttt{[link-rstudio]}. Bitte mache dies einfach schon im Vorfeld, damit wir Donnerstag direkt starten können. Du musst nichts weitermachen, nur diesen Anmeldeschritt. Alles Weitere erklären wir dann Schritt für Schritt am Donnerstag.
    \item \textbf{Gitlab}: Bitte logge dich hier \texttt{[link-gitlab]} einmalig mit deiner JLU-Kennung ein. Auch hier musst du nichts weiter machen, auch das erklären wir Schritt für Schritt am Donnerstag.
	\item \textbf{ILIAS}/Stud.IP: Du bist in Stud.IP bereits hinzugefügt. Von dort kommst du über die Schnittstelle direkt zum ILIAS-Kurs. Im ILIAS Kurs findest du die Information auch noch einmal. 
    \item \textbf{Gruppenzuteilung}: Für die Gruppenarbeiten kannst du bis Mittwochnacht zwei Prioritäten abgeben (automatisierter Zuteilungsmechanismus nach Präferenzen), in welcher der fünf Gruppen du arbeiten willst. Personen, die sich bis dahin nicht zuordnen, werden dann nach der automatisierten Verteilung von mir einer Gruppe zugeteilt.
    \item \textbf{Laptop/Tablet}: Wie schon in der Veranstaltungsankündigung genannt, bringe bitte deinen Laptop (oder Tablet mit externer Tastatur, sonst wird es anstrengend) mit, auf dem einer der gängigen Browser (Firefox, Chrome oder Safari) installiert ist. 
\end{enumerate}

Dieser Kurs wurde neu konzipiert und komplett neu erstellt. Das heißt, dies ist der erste Durchgang dieses Kurs. Da wir QSL-gefördert sind, sind wir verpflichtet den Kurs über das Semester zu evaluieren. Dazu gibt es eine Panelstudie, die dich in drei Abständen (vor, während, nach dem Kurs) zu deiner Lernmotivation und Lerneinstellungen befragt. Bitte fülle die erste Welle nun vor Donnerstag aus, bevor wir mit dem Kurs beginnen (dauert ca. 5 Minuten). Den Link dazu findest du hier: \texttt{[link-survey]}

Ich freue mich auf die Präsenzlehre und bis Donnerstag