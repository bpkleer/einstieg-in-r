\documentclass[11pt,a4paper]{article}
\usepackage[utf8]{inputenc}

\usepackage{xunicode}
\usepackage{xltxtra}
\usepackage{polyglossia}
\setdefaultlanguage{german}
\usepackage{lmodern}
\usepackage{csquotes}

%Einstellung Seitenränder
\usepackage[left=2cm,right=2cm,top=2cm,bottom=2cm]{geometry}

% math-umgebung
\usepackage{amsmath}
\usepackage{amsfonts}
\usepackage{amssymb}

% Color und Hyperlink packages
\usepackage{hyperref}
\usepackage[svgnames,hyperref]{xcolor}

% Datumspaket
\usepackage[german]{isodate}

% Table packages
\usepackage{booktabs}
\usepackage{longtable}

% Code highlighting
\usepackage{color}

\definecolor{dkgreen}{rgb}{0,0.6,0}
\definecolor{gray}{rgb}{0.5,0.5,0.5}
\definecolor{mauve}{rgb}{0.58,0,0.82}

\usepackage{listings}
\lstset{frame=tb,
basicstyle=\ttfamily,
language=R,
aboveskip=6mm,
belowskip=6mm,
showstringspaces=false,
columns=flexible,
numbers=none,
keywordstyle=\color{blue},
numberstyle=\tiny\color{gray},
commentstyle=\color{dkgreen},
stringstyle=\color{mauve},
breaklines=true,
breakatwhitespace=true,
tabsize=3
}
% Bibtex Einstellungen
\usepackage[%defernumbers=true,
%			%sorting =ydnt,
			backend=biber,
%			citestyle=apa,
%			bibstyle=apa, %apa funktioniert nicht irgendwie
			style=apa,
			%isbn=false,
			%block=space,
			%doi=true,
			%url=true,
			%maxnames=9
			]{biblatex}
\DeclareLanguageMapping{german}{german-apa}	

%Einstellungen hyperlink
\hypersetup{
    colorlinks=true,
    filecolor=Purple,    
    linkcolor = Orange,  
    urlcolor=MediumSeaGreen,
    citecolor = black,
    pdftitle={Seminarplan},
    pdfpagemode=FullScreen,
}

\urlstyle{same}
% how to use Hyperlinks: https://de.overleaf.com/learn/latex/Hyperlinks

%New Font
\usepackage{fontspec} % changing font (unten freimachen für Änderung
	% \defaultfontfeatures{Mapping = tex-text}
	\setmainfont{Fira Sans} %user-defined Font.

% Name, Titel, etc.
\author{B. Philipp Kleer}
\title{%
  Installationsanleitung git \\
%  \large Seminarplan \\
%  WiSe/SoSe
  }

\author{B. Philipp Kleer}
\date{\today \\ \small{Version: v1}}

\begin{document}

\maketitle

\textbf{Git} ist eine Version-Control von Dateien, es können also Veränderungen von jeder mitarbeitenden Person nachvollzogen werden. \textbf{Gitlab} liefert eine Online-Plattform zum Verwalten und Bearbeiten. An der JLU gibt es eine eigene \href{https://gitlab.ub.uni-giessen.de}{Distribution}.

Sofern man eine lokale Installation von RStudio nutzt, muss man zusätzlich noch \textbf{Git} auf dem Rechner installieren. Hierfür gibt es eine englischsprachige Anleitung (\href{https://happygitwithr.com/install-git.html}{Link}). Die Teilschritte werden aber auch hier erläutert. 

Wenn Sie nur die \textbf{RStudio Cloud} nutzen, müssen Sie \textbf{Git} nicht mehr installieren, da es bereits in der Cloud vorinstalliert ist.

\section*{Installation}
Für alle Benutzer von Windows, MacOS oder Linux steht unter folgendem \href{http://git-scm.com/downloads}{Link} eine Installationsdatei zur Verfügung. Alternativ ist die Installation per \textit{Commandline} in allen Betriebssystemen (Terminal/Eingabeaufforderung) möglich. Dazu bitte dem oben genanntem Link folgen. 

\section*{Nutzung von git}
Wie in der Präsentation genannt, können \textit{Git-Repositories} entweder über einen \textbf{HTTPS}- oder \textbf{SSH}-Zugriff angesprochen werden. \textbf{Wichtig:} Die \href{https://gitlab.ub.uni-giessen.de}{JLU-Distribution} ist per \textbf{SSH}-Zugriff nur aus dem VPN erreichbar, wir \textbf{empfehlen} daher die \textbf{HTTPS}-Verbindung zu nutzen, die auch ohne VPN funktioniert. 

\subsection*{Zugriff per HTTPS}
Wenn man auf die Repositories per \textbf{HTTPS} zugreift, muss man jedes Mal seinen Nutzernamen und Passwort eingeben. Die Eingabe des Nutzernamens kann man umgehen, in dem man einmalig in RStudio oder in der \textit{Commandline} seine Benutzerdaten festlegt. 

\subsubsection*{Einstellung in RStudio}
\begin{lstlisting}[language=R]
## install.packages("usethis")

library("usethis")
use_git_config(
	user.name = "Jane Doe", #hier den eigenen Namen hinschreiben
	user.email = "jane@example.org" #hier die E-Mail-Adresse hinschreiben, die in gitlab hinterlegt ist
)
\end{lstlisting}

\subsubsection*{Einstellung über Terminal/Eingabeaufforderung}
\begin{lstlisting}[language=bash]
git config --global user.name 'Jane Doe'
git config --global user.email 'jan@example.org'
git config --global --list
\end{lstlisting}

Anschließend kann man über den \textbf{HTTPS}-Zugriff die Git-Repositories lokal \textit{clonen}. 

\subsection*{Zugriff per SSH}
Der Zugriff per \textbf{SSH} ist ebenfalls möglich, aber nur aus dem Universitäts-VPN. Wir empfehlen daher den Zugriff per HTTPS.

Um den \textbf{SSH} Zugriff zu nutzen, muss man zuerst einen \textbf{SSH-Key} erstellen. Einen Key kann man einfach in RStudio erstellen. Dort geht man im Menü einfach auf \texttt{Tools -> Global Options -> Git -> Create RSA Key}. Wenn man einen Key über die \textit{Commandline}, erstellen will, hilft dieser \href{https://www.heise.de/tipps-tricks/SSH-Key-erstellen-so-geht-s-4400280.html}{Link}.

Anschließend klickt man auf \textbf{View Key} und kopiert den Key-Text. 

Dann loggt man sich im \href{https://gitlab.ub.uni-giessn.de}{JLU-Gitlab} unter seinem Profil ein und geht auf \texttt{Einstellungen -> SSH Keys} und kopiert den Text in das große Feld.

Dann gibt man noch einen Namen für diesen Key zur Identifikation ein und klickt auf \textbf{Add Key}. Ggbfs. gibt man ein Ablaufdatum des Keys ein. Jetzt ist alles fertig eingestellt.

Anschließend geht man über den Browser in das entsprechende Projekt und wählt bei Clonen den Link neben SSH aus.

\end{document}