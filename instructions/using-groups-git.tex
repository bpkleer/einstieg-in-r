\documentclass[11pt,a4paper]{article}
\usepackage[utf8]{inputenc}

%%\usepackage{xunicode}
%\usepackage{xltxtra}
\usepackage{polyglossia}
\setdefaultlanguage{german}
\usepackage{lmodern}
\usepackage{csquotes}

%Einstellung Seitenränder
\usepackage[left=2cm,right=2cm,top=2cm,bottom=2cm]{geometry}

% math-umgebung
\usepackage{amsmath}
\usepackage{amsfonts}
\usepackage{amssymb}

% Color und Hyperlink packages
\usepackage{hyperref}
\usepackage[svgnames,hyperref]{xcolor}

% Table packages
%\usepackage{booktabs}
%\usepackage{longtable}

% Code highlighting
\usepackage{color}

\definecolor{dkgreen}{rgb}{0,0.6,0}
\definecolor{gray}{rgb}{0.5,0.5,0.5}
\definecolor{mauve}{rgb}{0.58,0,0.82}

\usepackage{forest}
\usepackage{listings}
\lstset{frame=tb,
basicstyle=\ttfamily,
language=R,
aboveskip=6mm,
belowskip=6mm,
showstringspaces=false,
columns=flexible,
numbers=none,
keywordstyle=\color{blue},
numberstyle=\tiny\color{gray},
commentstyle=\color{dkgreen},
stringstyle=\color{mauve},
breaklines=true,
breakatwhitespace=true,
tabsize=3
}

%Einstellungen hyperlink
\hypersetup{
    colorlinks=true,
    filecolor=Purple,    
    linkcolor = Orange,  
    urlcolor=MediumSeaGreen,
    citecolor = black,
    pdftitle={Seminarplan},
    pdfpagemode=FullScreen,
}

\urlstyle{same}
% how to use Hyperlinks: https://de.overleaf.com/learn/latex/Hyperlinks

%New Font
\usepackage{fontspec} % changing font (unten freimachen für Änderung
	% \defaultfontfeatures{Mapping = tex-text}
	\setmainfont{Fira Sans} %user-defined Font.

% Name, Titel, etc.
\author{B. Philipp Kleer}
\title{%
  Gebrauchsanweisung Gruppen in gitlab \\
%  \large Seminarplan \\
%  WiSe/SoSe
  }

\author{B. Philipp Kleer}
\date{\today \\ \small{Version: v1}}

\begin{document}

In \textbf{gitlab} können Gruppen zur Organisation genutzt werden. Für die Nutzung in der Lehre bietet sich dies insbesondere an, da in Gruppen nicht nur mehrere Repositories gebündelt werden können, sondern auch \textbf{Subgroups} für Gruppenarbeiten angelegt werden können. Mit \textbf{gitlabber} können dann auch immer alle Subgruppen und alle Repositories einer Gruppe gepullt werden (siehe Anleitung gitlabber).

Die gesamte Gruppe ist wie folgt aufgebaut:

\begin{forest}
  for tree={
    font=\ttfamily,
    grow'=0,
    child anchor=west,
    parent anchor=south,
    anchor=west,
    calign=first,
    edge path={
      \noexpand\path [draw, \forestoption{edge}]
      (!u.south west) +(7.5pt,0) |- node[fill,inner sep=1.25pt] {} (.child anchor)\forestoption{edge label};
    },
    before typesetting nodes={
      if n=1
        {insert before={[,phantom]}}
        {}
    },
    fit=band,
    before computing xy={l=15pt},
  }
[intro-r-fall21 (Kurs, \textit{Main group})
  [course materials (\textit{project})]
  [effie (Gruppe 1, \textit{subgroup 1})
    [lab tasks (\textit{project})]
  ]
  [gale (Gruppe 2, \textit{subgroup 2})
    [lab tasks (\textit{project})]
  ]
  [katniss (Gruppe 3, \textit{subgroup 3})
    [lab tasks (\textit{project})]
  ]
  [peeta (Gruppe 4, \textit{subgroup 4})
    [lab tasks (\textit{project})]
  ]
  [rue (Gruppe 5, \textit{subgroup 5})
    [lab tasks (\textit{project})]
  ]
]
\end{forest}

\section{Rechtezuteilungen}

\subsection{Hauptgruppe}
In der Hauptgruppe werden alle Kursteilnehmer:innen als \textbf{Developer} eingetragen. In den Einstellungen ist eingestellt, dass nur \textbf{Maintainer} Projekte erstellen können. In den Einstellungen der Gruppe haben somit Kursteilnehmer:innen nur Leserechte der Repository in der Hauptgruppe und keine Schreibrechte. 

Wir haben uns dazu entschieden, nur Leserechte einzustellen, da dies der erste Kontakt mit gitlab ist und wir vermeiden möchten, dass Studierende das zentrale Repository \textbf{course-materials} zerschießen. In fortgeschrittenen Kursen kann es durchaus sinnvoll sein, höhere Rechte zu geben, da so Studierende Fehler in den Kursmaterialien verbessern und pushen könnten. 

\subsection{Subgroups}
Automatisch haben in gitlab alle Personen in \textbf{Subgroups} die Rechte von der übergeordneten Gruppe, also hier alle \textbf{Developer}-Status. 

In den Subgruppen wollen wir, dass die Personen aus der jeweiligen Gruppe auch an den Repositories arbeiten können. Daher muss man jede Person aus der Gruppe in der \textbf{Subgroup} erneut einladen und zwar als \textbf{Maintainer}. Dies gilt ebenfalls für das \textbf{lab tasks}-Repository (erneut einladen mit \textbf{Maintainer}-Status. Wenn das nicht geschieht, können die Studierenden in RStudio nicht ihre Änderungen pushen. 

\section{Implementation in RStudio Cloud}
In diesem Kurs arbeiten wir mit einer RStudio Cloud Lizenz, in dem wir einen \textbf{Workspace} vorbereitet haben, in den die Studierenden per SSO-Login sich einwählen können. Zur Implementation von git gibt es z.B. hier [Hilfe](https://www.geo.uzh.ch/microsite/reproducible_research/post/rr-rstudio-git/).

Der \textbf{Workspace} ist wie folgt aufgebaut:

\begin{forest}
  for tree={
    font=\ttfamily,
    grow'=0,
    child anchor=west,
    parent anchor=south,
    anchor=west,
    calign=first,
    edge path={
      \noexpand\path [draw, \forestoption{edge}]
      (!u.south west) +(7.5pt,0) |- node[fill,inner sep=1.25pt] {} (.child anchor)\forestoption{edge label};
    },
    before typesetting nodes={
      if n=1
        {insert before={[,phantom]}}
        {}
    },
    fit=band,
    before computing xy={l=15pt},
  }
[Stats$^+$: Einstieg in R (Kurs, \textit{Workspace})
  [course materials (\textit{project})]
  [effie (Gruppe 1, \textit{project})]
  [gale (Gruppe 2, \textit{project})]
  [katniss (Gruppe 3, \textit{project})]
  [peeta (Gruppe 4, \textit{project})]
  [rue (Gruppe 5, \textit{project})]
]
\end{forest}

Jede Gruppe hat ein eigenes Projekt. Wie es in RStudio Cloud üblich ist, können Studierende die vorbereiteten Projekte aufrufen und dann eine lokale Kopie speichern (\textbf{course-materials} und die entsprechende Gruppe).

In den einzelnen Projekten haben wir die git-Verknüpfung vorbereitet, damit Studierende diesen Schritt überspringen können. 

Dazu initiiert man zuerst \textbf{git} im entsprechenden Projekt über \begin{verbatim}Tools -> Version Control -> Project Setup \end{verbatim}. Danach startet sich dann das Projekt neu. 

Im Terminal muss man dann nur die folgenden Befehle eingeben, um das entsprechende Repository zu verbinden:

\begin{lstlisting}[language=bash]
git remote add origin https://gitlab.com/myrepo.git

git pull origin main

git branch --set-upstream-to=origin/main master
\end{lstlisting}

Um dann git nutzen zu können, müssen die einzelnen Kursteilnehmer:innen nur ihren Git-Name und E-Mail-Adresse fürs committen hinzufügen (wir nutzen R und die \textit{library usethis}). Beim Pullen oder Pushen muss dann jeweils der Benutzername und das Passwort aus \textbf{gitlab} eingegeben werden (Studierendenkennung).

\begin{lstlisting}[language=R]
library("usethis")

use_git_config( 
  user.name = "Jane Doe", #hier deinen Namen eingeben
  user.email = "jane@example.org" # Hier deine JLU-E-Mail-Adresse eingeben
)
\end{lstlisting}

Anschließend funktioniert \textbf{pullen} und \textbf{pushen} entsprechend der Rechte in \textbf{gitlab}. Im Repository \textbf{course-materials} können Studierende nur pullen, in den \textbf{lab tasks} auch pushen. 

\textbf{Wichtig:} Auch im Repository \textbf{course-materials} müssen Kursteilnehmer:innen ihre Änderungen (also z.B. Anmerkungen) committen, da diese sonst beim pullen verloren gehen. 


\end{document}
