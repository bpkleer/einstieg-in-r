\documentclass[11pt,a4paper]{article}
\usepackage[utf8]{inputenc}

\usepackage{xunicode}
\usepackage{xltxtra}
\usepackage{polyglossia}
\setdefaultlanguage{german}
\usepackage{lmodern}
\usepackage{csquotes}

%Einstellung Seitenränder
\usepackage[left=2cm,right=2cm,top=2cm,bottom=2cm]{geometry}

% math-umgebung
\usepackage{amsmath}
\usepackage{amsfonts}
\usepackage{amssymb}

% Color und Hyperlink packages
\usepackage{hyperref}
\usepackage[svgnames,hyperref]{xcolor}

% Datumspaket
\usepackage[german]{isodate}

% Table packages
\usepackage{booktabs}
\usepackage{longtable}

% Code highlighting
\usepackage{color}

\definecolor{dkgreen}{rgb}{0,0.6,0}
\definecolor{gray}{rgb}{0.5,0.5,0.5}
\definecolor{mauve}{rgb}{0.58,0,0.82}

\usepackage{listings}
\lstset{frame=tb,
basicstyle=\ttfamily,
language=bash,
aboveskip=6mm,
belowskip=6mm,
showstringspaces=false,
columns=flexible,
numbers=none,
keywordstyle=\color{blue},
numberstyle=\tiny\color{gray},
commentstyle=\color{dkgreen},
stringstyle=\color{mauve},
breaklines=true,
breakatwhitespace=true,
tabsize=3
}
% Bibtex Einstellungen
\usepackage[%defernumbers=true,
%			%sorting =ydnt,
			backend=biber,
%			citestyle=apa,
%			bibstyle=apa, %apa funktioniert nicht irgendwie
			style=apa,
			%isbn=false,
			%block=space,
			%doi=true,
			%url=true,
			%maxnames=9
			]{biblatex}
\DeclareLanguageMapping{german}{german-apa}	

%Einstellungen hyperlink
\hypersetup{
    colorlinks=true,
    filecolor=Purple,    
    linkcolor = Orange,  
    urlcolor=MediumSeaGreen,
    citecolor = black,
    pdftitle={Seminarplan},
    pdfpagemode=FullScreen,
}

\urlstyle{same}
% how to use Hyperlinks: https://de.overleaf.com/learn/latex/Hyperlinks

%New Font
\usepackage{fontspec} % changing font (unten freimachen für Änderung
	% \defaultfontfeatures{Mapping = tex-text}
	\setmainfont{Fira Sans} %user-defined Font.

% Name, Titel, etc.
\author{B. Philipp Kleer}
\title{%
  Anwendung gitlabber \\
%  \large Seminarplan \\
%  WiSe/SoSe
  }

\author{B. Philipp Kleer}
\date{\today \\ \small{Version: v1}}

\begin{document}

\maketitle

\textbf{Gitlabber} ist eine Software, mit der Gruppen aus \textbf{git} gleichzeitig gecloned werden können. Es ermöglicht also der Dozent:in eine einfache Handhabung, wenn Code aus Gruppen ausgelesen werden soll. 

Die Dokumentation zur Installation findet sich \href{https://github.com/ezbz/gitlabber}{\textbf{hier}}.

Um \textbf{gitlabber} zu nutzen, benötigst du einen Personal Access Token (PAT) in gitlab, der sowohl Schreib- als auch Leserechte auf deiner API hat. Dazu gehe unter deinem Profil auf Einstellungen -> Personal Access Token und erstelle einen spezifischen Token, falls du noch keinen hast. 

Wenn du den PAT erstellt hast, kannst du in den Terminal wechseln: Hier lässt du dir erstmal eine Ansicht anzeigen:

\begin{lstlisting}[language=bash]
gitlabber -t yoursecrettoken -u https://gitlab.ub.uni-giessen.de/ -p
\end{lstlisting}

Du siehst dann einen Pfadbaum für alle deine Zugriffe in gitlab.

Du willst jetzt aber nur den einen Kurs clonen, hier bei uns ist das \texttt{intro-r-spring22}. Wir schränken daher den Zugriff auf diese Gruppe ein (\texttt{-i "/..."}), wir greifen per \texttt{https}-Zugriff auf gitlab zu (Standard ist SSH bei gitlabber) (\texttt{-m 'http'}) und speichern es in dem Folder, in dem wir gerade sind (\texttt{./}):
\begin{lstlisting}[language=bash]
gitlabber -t yoursecrettoken -u https://gitlab.ub.uni-giessen.de/ -i '/intro-r-spring22**' -m 'http' ./
\end{lstlisting}

Danach sind alle Subgruppen und Projekte der Gruppe \texttt{intro-r-spring22} lokal gecloned. Du kannst dann einzeln (z. B. nach einer Korrektur) zurückpushen. 

\end{document}