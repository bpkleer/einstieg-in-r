\documentclass[12pt,a4paper]{article}
\usepackage[utf8]{inputenc}
\usepackage[german]{babel}
\usepackage[T1]{fontenc}
\usepackage[left=1.5cm,right=1.5cm,top=2cm,bottom=2cm]{geometry}
\usepackage{article-german}

% Code highlighting
%\usepackage{minted}
%\usemintedstyle{monokai}
%\setminted{bgcolor=black}
%\setminted{linenos = true}

% Dateibaum
\usepackage{forest}

%New Font
\usepackage{fontspec} % changing font (unten freimachen für Änderung
	% \defaultfontfeatures{Mapping = tex-text}
	\setmainfont{Fira Sans} %user-defined Font.

\title{Lab 3 \\ \normalsize \textit{Lernblock 3}}
\author{Meike Schulz-Narres, B. Philipp Kleer}
\date{\today \\ \small{Version: 0.1}} 

\begin{document}
\maketitle

\begin{enumerate}
	\item Erstelle eine Tabelle \texttt{table1} der Variable \texttt{stfeco} aus dem PSS. Die Tabelle soll neben den gültigen Fällen auch die fehlenden Werte anzeigen.
	\item Erstellt eine Kreuztabelle \texttt{table2} aus den Variablen \texttt{district} und \texttt{income} aus dem PSS.
	\item Nutze das Objekt \texttt{table2} und gib nun die Spaltenprozente an. Speichere dies im Objekt \texttt{table2col}.
	\item Teste den Zusammenhang zwischen den Variablen \texttt{agea} und \texttt{lrscale}. 
	\begin{enumerate}
		\item Erstelle zuerst ein Scatterplot und speichere dies in \texttt{plot1}. 
		\item Überlege dir, welches Maß du anwenden kannst!
		\item Berechne das Maß und speichere das Ergebnis in \texttt{res1}. Verwende hierzu die \textit{library} \texttt{psych}.
	\end{enumerate}
	\item Du möchtest wissen, ob sich das Vertrauen der Befragten in das Parlament (\texttt{trstprl}) zwischen den Geschlechtern (\texttt{gndr}) unterscheidet. Du hast dazu in einem angesehenen Journal einen Beitrag gelesen, dass Männer mehr Vertrauen als Frauen haben. Daher testest du einseitig! Speichere den Test im Objekt \texttt{mt1}!
\end{enumerate}

\end{document}