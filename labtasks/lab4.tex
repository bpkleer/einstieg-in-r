\documentclass[12pt,a4paper]{article}
\usepackage[utf8]{inputenc}
\usepackage[german]{babel}
\usepackage[T1]{fontenc}
\usepackage[left=1.5cm,right=1.5cm,top=2cm,bottom=2cm]{geometry}
\usepackage{article-german}

% Code highlighting
%\usepackage{minted}
%\usemintedstyle{monokai}
%\setminted{bgcolor=black}
%\setminted{linenos = true}

% Dateibaum
\usepackage{forest}

%New Font
\usepackage{fontspec} % changing font (unten freimachen für Änderung
	% \defaultfontfeatures{Mapping = tex-text}
	\setmainfont{Fira Sans} %user-defined Font.

\title{Lab 4 \\ \normalsize \textit{Lernblock 4}}
\author{Leon Klemm, B. Philipp Kleer}
\date{\today \\ \small{Version: 0.1}} 



\begin{document}
\maketitle

\begin{enumerate}
	\item Lade den Datensatz \texttt{pss} in das \textit{environment}.		
	\item Berechne folgende lineare Regressionsmodelle auf die Variable Vertrauen in Politiker:innen (\texttt{trstplt}):
	\begin{enumerate}
		\item \texttt{model1}: Zufriedenheit mit der Demokratie (\texttt{stfdem})
		\item \texttt{model2}: plus Vertrauen ins Parlament (\texttt{trstprl})
		\item \texttt{model3}: plus Vertrauen in Parteien (\texttt{trstprt})
		\item \texttt{model4}: plus Geschlecht (\texttt{gndr}), Schulabschluss (\texttt{edu}), Alter (\texttt{agea}) und Einkommen (\texttt{income}) als Kontrollvariablen. Referenzkategorien: \texttt{weiblich}, \texttt{Master}, \texttt{10. Dezil}
	\end{enumerate}
	\item Nun extrahiere aus den Modellen Werte und speichere diese jeweils in einem Objekt:
	\begin{enumerate}
		\item erklärte Varianz von \texttt{model2} in \texttt{rsquaredmod2}
        \item erklärte Varianz von \texttt{model4} in \texttt{rsquaredmod4}
        \item Namen aller signifikanten Variablen des \texttt{model3} in Charakter-Vektor \texttt{sigvars}
	\end{enumerate}
	 \item Gib einen einfachen Koeffizienten-Plot aus, in dem die Modelle \texttt{model1}, \texttt{model2} und \texttt{model3} angezeigt werden, und speichere diesen Plot in \texttt{olsplot}. Nutze das Paket \texttt{dotwhisker}.
	 \item Speichere eine einfache exportfähig Tabelle, in der alle Modelle aufgelistet sind, im Objekt \texttt{olstab}. Nutze das Paket \texttt{modelsummary}.
\end{enumerate}

\end{document}