\documentclass[12pt,a4paper]{article}
\usepackage[utf8]{inputenc}
\usepackage[german]{babel}
\usepackage[T1]{fontenc}
\usepackage[left=1.5cm,right=1.5cm,top=2cm,bottom=2cm]{geometry}
\usepackage{article-german}

% Code highlighting
%\usepackage{minted}
%\usemintedstyle{monokai}
%\setminted{bgcolor=black}
%\setminted{linenos = true}

% Dateibaum
\usepackage{forest}

%New Font
\usepackage{fontspec} % changing font (unten freimachen für Änderung
	% \defaultfontfeatures{Mapping = tex-text}
	\setmainfont{Fira Sans} %user-defined Font.

\title{Lab 2 \\ \normalsize \textit{Lernblock 2}}
\author{Leon Klemm, B. Philipp Kleer}
\date{\today \\ \small{Version: 0.1}} 



\begin{document}
\maketitle

	\begin{enumerate}
		\item Lade den Datensatz \texttt{pss} in das \textit{environment}. Wie viele Beobachtungen und Variablen hat der Datensatz? Speichere die Anzahl der Variablen in das Objekt \texttt{varCount} und die Anzahl der Beobachtungen in das Objekt \texttt{varCases}.
		\item Gib die deskriptive Statistik für die Variable \textit{Vertrauen in Politiker} aus. Speichere den Wert jeweils in einem Objekt! Der Objektname steht in Klammern. Schaue dir vorher im Codebook an, welches Skalenniveau die Variable hat und wie sie kodiert ist. Beschreibe innerhalb des Skripts die deskriptiven Werte dieser Variable über die Kommentarfunktion.
    	\begin{enumerate}
    		\item[a)] Minimum (\texttt{tpMin}) \& Maximum (\texttt{tpMax})
    		\item[b)] Quartile (\texttt{tpQuart}) \& IQR (\texttt{tpIQR})
    		\item[c)] arithmetisches Mittel (\texttt{tpMean}) und Median (\texttt{tpMed})
    		\item[d)] Varianz (\texttt{tpVar}) \& Standardabweichung (\texttt{tpSd})
    	\end{enumerate}
		\item Gib die Quintile für die Variable Vertrauen in die Justiz aus. Speichere das Ergebnis im Objekt (\texttt{tjQuint})
		\item Gib die durchschnittlichen Arbeitsstunden der Menschen pro Woche je Distrikt an (\texttt{wkDist}).
		\item Erstelle ein Subset, welches Personen mit sehr niedrigem Bildungsabschluss (ES-ISCED I) und geringem Einkommen (1. - 3. Dezil) umfasst. Speichere das Subset im Objekt \texttt{dfLiLe}.
\end{enumerate}

\textbf{Zusatzaufgabe Programmierexkurs (freiwillig)}:
\begin{enumerate}
	\item[6.] Nutze dein Wissen zu einer \textit{for}-Schleife und \textit{If-Else}-Bedingungen, um eine neue Variable zu schaffen, die anzeigt, ob eine Person unter 25 Jahre alt ist. Die neue Variable soll Teil des Datensatzes \texttt{pss} sein und den Namen \texttt{ageaRec} haben.
\end{enumerate}

\end{document}